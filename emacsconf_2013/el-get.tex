\documentclass{beamer}

\usepackage{minted}
\usemintedstyle{emacs}
\usepackage{beamerthemesplit}
\usepackage[utf8]{inputenc}
%% \usetheme{AnnArbor}
\usetheme{Boadilla}
%% \usetheme{Pittsburgh}
%% \usecolortheme{beaver}
\beamertemplatetransparentcovered

\title{El-Get}
\subtitle{\texttt{M-x apt-get}}
\author{Dimitri Fontaine \url{dim@tapoueh.org}}
\date{March, 30 2013}

\begin{document}

\frame{\titlepage}

\section{Introduction}

\begin{frame}[fragile]
  \frametitle{Dimitri Fontaine}

  \begin{center}
    \textbf{2ndQuadrant France}
    \linebreak
    PostgreSQL Major Contributor
    \linebreak
    Emacs Lisp Hacker
  \end{center}
  \vfill

\begin{columns}[c]
\column{.75\textwidth} 

  \begin{itemize}
   \item \texttt{pgloader}, \texttt{prefix}, \texttt{skytools}, \texttt{debian}, …
   \item \texttt{\textbf{CREATE EXTENSION}}
   \item \texttt{\textbf{CREATE EVENT TRIGGER}}
   \item \textit{El-Get}
   \item \textit{Emacs Lisp, Common Lisp}
  \end{itemize}  

\column{.25\textwidth}
\begin{center}
  \includegraphics[height=5em]{bulle-blue-icon.png}
\end{center}
\end{columns}
\end{frame}

\section{before el-get}

\begin{frame}[fragile]

  \center{\Huge{Emacs: Emacs Lisp}}

\end{frame}

\begin{frame}
  \frametitle{How I used to do things}
  
  \center{\texttt{C-x C-/ runs the command goto-last-change}}
  \vfill

  \begin{itemize}
  \item Talk about it on IRC
  \item Search for it in \url{http://emacswiki.org}
  \item \texttt{wget} the file
  \item Somewhere in your \texttt{load-path}
  \item Beware of whole directories vs single file
  \item Read installation notes, edit your setup to \texttt{require} it
  \item Look for docs (\texttt{.info}) and install those yourself
  \end{itemize}  
\end{frame}

\begin{frame}
  \frametitle{How I used to do things}
  
  \center{\texttt{C-x C-/ runs the command goto-last-change}}
  \vfill

  \begin{itemize}
  \item Now try it!
  \item If you like it, you're about done (\textit{customizing})
  \item Just think about upgrading to a newer version, if needed
  \item (e.g. to support a newer Emacs)
  \item If you don't like it, undo all the previous bits
  \end{itemize}  
\end{frame}

\begin{frame}
  \frametitle{Managing Emacs Extensions}
  
  \center{Installing is but the first step}
  \vfill

  \begin{itemize}
  \item Version Control as if it were yours
  \item Sharing to other machines
  \end{itemize}  
\end{frame}

\begin{frame}
  \frametitle{How much can you install and manage this way?}
  
  \center{You tend to only install \textit{trusted} extensions}

\end{frame}

\section{el-get}

\begin{frame}[fragile] 
  \begin{center}
    \Huge{El-Get}
    \vfill

    \includegraphics[height=7em]{el-get-logo.png}
  \end{center}
\end{frame}

\begin{frame}
  \frametitle{Think \texttt{apt-get} for Emacs}

  \center{Social Package Management System}
  \vfill

\begin{columns}
\column{.75\textwidth} 

  \begin{itemize}
   \item \texttt{M-x el-get-list-packages}
   \item \texttt{M-x el-get-describe}
   \item \texttt{M-x el-get-install}
   \item \texttt{M-x el-get-update}, \texttt{M-x el-get-reload}
   \item \texttt{M-x el-get-remove}
   \item \texttt{M-x el-get-self-update}
  \end{itemize}

\column{.25\textwidth}
  \begin{center}
    \includegraphics[height=7em]{el-get-logo.png}
  \end{center}
\end{columns}
\end{frame}

\begin{frame}
  \frametitle{Features}

  \center{\textit{moar} please}
  \vfill

  \begin{itemize}
   \item Package dependencies
   \item Cleanup command (\textit{autoremove})
   \item Extensive Info Documentation, with a Glossary!
   \item Stable branch is very stable...
   \item Easy install
   \item Good default setup, more complex setup supported
   \item Easy per-package setup (\texttt{el-get-user-package-directory})
   \item Autoloads
  \end{itemize}
\end{frame}

\begin{frame}
  \frametitle{Features}

  \center{\textit{moar} please}
  \vfill

  \begin{itemize}
   \item Checksum support
   \item git: Shallow clones
   \item git: Submodules
   \item Per \texttt{system-type} build commands
   \item Windows support (users called) (yes they still exist)
   \item Build commands, \texttt{byte-compile}
   \item Recipe Checker
  \end{itemize}
\end{frame}

\begin{frame}
  \frametitle{Benefits}

  \center{What happens at \texttt{el-get-install} time?}
  \vfill

  \begin{itemize}
   \item El-Get knows how to fetch the extension
   \item Even if it's more than a single file
   \item The \textit{recipe} contains the necessary installation bits
   \item Docs get installed and integrated into your \texttt{C-h i}
   \item The package is \textit{byte-compiled}
   \item Then \texttt{autoload}ed
   \item If the \textit{recipe} says so, features are \texttt{require}d
  \end{itemize}
\end{frame}

\begin{frame}
  \frametitle{Setup}

  \center{Hey it's Emacs, I want to do it my own way!}
  \vfill

  \begin{itemize}
   \item Initializing a package involves custom actions
   \item \texttt{el-get-sources}
   \item \texttt{:before}, \texttt{:after}, \texttt{:features}
   \item \texttt{el-get-user-package-directory}, \texttt{init-pkgname.el}
   \item \texttt{M-x el-get-init}
  \end{itemize}
\end{frame}

\section{recipes}

\begin{frame}[fragile]
  \begin{center}
    \Huge{Recipes}
    %% \includegraphics[height=18em]{el-get.big.png}
  \end{center}
\end{frame}

\begin{frame}
  \frametitle{Gimme your recipes}

  \center{All your recipes are belong to us!}
  \vfill

  \begin{itemize}
   \item recipes are packaging script
   \item can be pretty simple
   \item can be pretty involved
   \item easy enough to make your own locally
   \item \texttt{el-get-sources} or in a local file
   \item require no edits of upstream sources
   \item find a recipe with \texttt{M-x el-get-find-recipe-file}
  \end{itemize}
\end{frame}

\begin{frame}[fragile]
  \frametitle{Recipe Example 1, el-get}

  \center{\texttt{el-get} is self-hosted and has a \texttt{*scratch*}
    installer} \vfill

  \begin{minted}{cl}
(:name el-get
 :website "https://github.com/dimitri/el-get#readme"
 :description "Manage the external elisp bits and pieces you depend upon."
 :type github
 :branch "4.stable"
 :pkgname "dimitri/el-get"
 :info    "."
 :load    "el-get.el")
  \end{minted}
\end{frame}

\begin{frame}[fragile]
  \frametitle{Recipe Example 2, escreen}

  Escreen is available on some website, not ELPA ready, Copyright 1992, 94,
  95, 97, 2001, 2005 Noah S. Friedman \vfill

  \begin{minted}{cl}
(:name escreen
 :description "Emacs window session manager"
 :type http
 :url "http://www.splode.com/~friedman/software/emacs-lisp/src/escreen.el"
 :post-init (lambda ()
              (autoload 'escreen-install "escreen" nil t)))
  \end{minted}
\end{frame}

\begin{frame}[fragile]
  \frametitle{Recipe Example 3}

  Solarized is a recent color-theme, and has been a fast moving target,
  hosted at github, not available as an ELPA package. \vfill

  \begin{minted}{cl}
(:name color-theme-solarized
 :description "Emacs highlighting using Ethan Schoonover's Solarized color scheme"
 :type github
 :pkgname "sellout/emacs-color-theme-solarized"
 :depends color-theme
 :prepare (progn
            (add-to-list 'custom-theme-load-path default-directory)
            (autoload 'color-theme-solarized-light "color-theme-solarized"
              "color-theme: solarized-light" t)
            (autoload 'color-theme-solarized-dark "color-theme-solarized"
              "color-theme: solarized-dark" t)))
  \end{minted}
\end{frame}

\frame{
  \frametitle{Sharing Recipes}

  \center{Distributed packaging: the Emacs Packages Social Network}
  \vfill

  \begin{center}
    Just email your recipe to your friend
  \end{center}
}

\frame{
  \frametitle{Sharing Recipes}

  \center{Distributed packaging: the Emacs Packages Social Network}
  \vfill

  \begin{itemize}
   \item Once tested, send them to me (github, mail)
   \item Once pushed, \texttt{el-get-self-update}
   \item Available in \texttt{M-x el-get-install}
   \item and in \texttt{M-x el-get-describe}
   \item and in \texttt{M-x el-get-list-packages}
  \end{itemize}
}

\frame{
  \frametitle{Where to get the packages from?}

  \center{El-Get has more than one \textit{method} to fetch your package}
  \vfill

  \begin{itemize}
   \item git (with branch support), git-svn
   \item github, github-tar, github-zip
   \item http, http-tar, http-zip, ftp
   \item emacswiki, \texttt{M-x el-get-emacswiki-refresh}
   \item emacsmirror
   \item svn, bzr, darcs, cvs, hg, fossil
   \item apt-get, pacman, fink, brew
   \item elpa (multi repositories, \texttt{:repo})
  \end{itemize}
}

\section{usage}

\begin{frame}[fragile]
  \begin{center}
    \Huge{Installing and Using El-Get}
    %% \includegraphics[height=18em]{el-get.big.png}
  \end{center}
\end{frame}

\begin{frame}[fragile]
  \frametitle{The \texttt{*scratch*} installer}

  \begin{minted}{cl}
;; So the idea is that you copy/paste this code into your *scratch* buffer,
;; hit C-j, and you have a working el-get.
(url-retrieve
 "https://github.com/dimitri/el-get/raw/master/el-get-install.el"
 (lambda (s)
   (end-of-buffer)
   (eval-print-last-sexp)))
  \end{minted}
\end{frame}

\begin{frame}[fragile]
  \frametitle{The initial setup in \texttt{.emacs}}

  \begin{minted}{cl}
(add-to-list 'load-path "~/.emacs.d/el-get/el-get")

(unless (require 'el-get nil 'noerror)
  (with-current-buffer
      (url-retrieve-synchronously
       "https://raw.github.com/dimitri/el-get/master/el-get-install.el")
    (goto-char (point-max))
    (eval-print-last-sexp)))

(el-get 'sync)
  \end{minted}
\end{frame}

\section{Community}

\begin{frame}[fragile]
  \begin{center}
    \Huge{Active Community}
    %% \includegraphics[height=18em]{el-get.big.png}
  \end{center}
\end{frame}

\frame{
  \frametitle{Some numbers, because we know you like them}

  \begin{itemize}
   \item 181 contributors
   \item 748 recipes included
   \item 1795 emacswiki files supported
   \item 3054+ emacsmirror repositories
   \item 7200+ lines of emacs-lisp fun (including 20\% of comments)
   \item 2+ years old, long time stable, production ready
  \end{itemize}
}

\section{Conclusion}

\frame{

\begin{center}
  \textbf{El-Get}. Any question?
  \linebreak
  \linebreak

  \begin{center}
    \includegraphics[height=2.1in]{el-get-logo.png}
  \end{center}
\end{center}
}

\end{document}
