%%% Skytools3 --- what's good and new, why you want to upgrade, how to plan
%%% your upgrade

\documentclass[english]{beamer}
\usepackage[utf8,latin9]{inputenc}
%%\usepackage[latin9]{inputenc}
\usepackage[T1]{fontenc}
\usepackage{babel}

\usepackage{beamerthemesplit}
\usetheme{Warsaw}
\beamertemplatetransparentcovered

\logo{\includegraphics[height=0.5cm]{2ndQuadrant-cross.png}}

\title{Skytools 3.0}
\author{Dimitri Fontaine, 2ndQuadrant France}
\date{July, 12 2011}

\begin{document}

\frame{\titlepage}

\section*{Agenda}
\frame{
  \frametitle{Content}
  \tableofcontents[pausesections]
}

\section{Skytools}

\subsection{History}

\frame{
  \frametitle{Where does Londiste comes from?}

  \begin{itemize}
   \item<1-> Slony-I
   \item<2-> Write load versus reliability
   \item<2-> Network Topology versus flexibility
   \item<3-> Only solves a subset of the problems
   \item<4-> Shared some internal implementation (as \texttt{txid\_snapshot})
  \end{itemize}
}

\subsection{Software Architecture}

\frame{
  \frametitle{Skytools Architecture}

  \begin{itemize}
   \item<1-> in-database \texttt{PGQ} modules (C coded triggers)
   \item<2-> Skytools python framework
   \item<3-> PGQ python librairies, utilities to code consumers (daemons)
   \item<4-> Specific consumers, such as \texttt{londiste}
   \item<5-> Extra tools based upon the framework, such as \texttt{walmgr}
  \end{itemize}
}

\frame{
  \frametitle{What does that mean for me?}

  A flexible architecture allows

  \begin{itemize}
   \item<1-> Cross version and architecture replication
   \item<1-> Version and Schema upgrade support
   \item<2-> General asynchronous processing support
   \item<3-> Limited set of concepts
  \end{itemize}
}

\section{Skytools 3.0}

\subsection{PGQ improvements}

\subsection{Replicating Queues}

\frame{
  \frametitle{PGQ Nodes}

  PGQ architecture has changed so that the queue themselves can be
  replicated from a node to another, to allow for online switchover
  facilities.  That comes with 3 node types:

  \begin{itemize}
   \item Root
   \item Branch
   \item Leaf
  \end{itemize}  
}

\begin{frame}[fragile]
  \frametitle{PGQ Nodes}

  \begin{example}[qadmin status]
\begin{verbatim}
bdd2$ londiste3 demo.ini status
Queue: demo   Local node: bdd2
 
bdd1 (root)
  |                           Tables: 85/0/0
  |                           Lag: 3s, Tick: 8706
  +--bdd2 (branch)
     |                        Tables: 85/0/0
     |                        Lag: 3s, Tick: 8706
     +--web1 (leaf)
     |                        Tables: 85/0/0
     |                        Lag: 3s, Tick: 8706
     +--web2 (leaf)
     |                        Tables: 85/0/0
     |                        Lag: 3s, Tick: 8706
     +--web3 (leaf)
                              Tables: 85/0/0
                              Lag: 3s, Tick: 8706
\end{verbatim}
  \end{example}
\end{frame}

\subsection{Cooperative Consumers}

\frame{
  \frametitle{Cooperative Consumers}

  PGQ is often used on its own, for general queuing needs that are not tied
  to replication.  In some cases you want to have multiple consumers
  draining in parallel from the same queue, but \textit{sharing} the work.
  \linebreak
  \pause
  
  Includes failover.  When a consumer stops consuming, its current batch
  gets reassigned to another one after a configurable timeout.
}

\subsection{Admin tool, \tettt{qadmin}}

\frame{
  \frametitle{\texttt{qadmin}}

  \texttt{qadmin} is the new interactive command line, replacing
  \texttt{pgqadm} and offering a much richer set of commands.

  \begin{itemize}
   \item connect, install
   \item create queue, alter queue, show ...
   \item register, unregister
   \item londiste commands
  \end{itemize}  
}

\subsection{Londiste improvements}

\frame{
  \frametitle{Londiste improvements}

  Londiste itself received some new features

  \begin{itemize}
   \item<1-> Execute scripts (\texttt{DDL} support)
   \item<1-> \texttt{londiste ... add --create}
   \item<1-> Parallel \texttt{COPY}
   \item<2-> Column granularity
   \item<3-> Online topology changes
   \item<3-> Custom handlers (\textit{plugins})
   \item<4-> Documentation...
  \end{itemize}
}

\subsection{\texttt{debian} Packaging}

\frame{
  \frametitle{\texttt{debian} Packaging}

  Completely new for \texttt{skytools3}

  \begin{itemize}
   \item skytools3
   \item python-pgq3
   \item python-skytools3
   \item skytools3-walmgr
   \item skytools3-ticker
   \item postgresql-8.4-pgq3
   \item postgresql-9.0-pgq3
  \end{itemize}
}

\section{Upgrading}

\subsection{Upgrading PGQ}

\frame{
  \frametitle{Upgrading PGQ}

  How to upgrade from \texttt{PGQ 2} to \texttt{PGQ 3}

  \begin{itemize}
   \item<1-> Best way, do it on a new target database
   \item<2-> Upgrade \texttt{PGQ} schema
   \item<3-> Upgrade modules
   \item<4-> Upgrade to \texttt{pgqd}
  \end{itemize}
}

\subsection{Upgrading Services}

\frame{
  \frametitle{Upgrading Services}

  That's the easy part (thanks to compatiblity)

  \begin{itemize}
   \item<1-> Daemons: the API is compatible
   \item<2-> Mostly: no more \texttt{failed} queue
   \item<3-> Londiste should run fine
  \end{itemize}  
}

\section{Conclusion}

\subsection{Any question?}

\frame{
  \frametitle{Any question?}

  \begin{center}
    Now is a pretty good time to ask!
  \end{center}
}

\end{document}
