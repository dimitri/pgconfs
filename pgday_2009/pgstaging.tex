\documentclass{beamer}
\usepackage[latin9]{inputenc}
\usepackage[T1]{fontenc}

\usepackage{beamerthemesplit}
\usetheme{Warsaw}
\beamertemplatetransparentcovered

\title{pg\_staging}
\author{Dimitri Fontaine}
\date{November, 7th 2009}

\begin{document}

\frame{\titlepage}

\section*{Plan}
\frame{
  \frametitle{How to manage your staging environments?}
  \tableofcontents
}

\section{Premise: you do have backups, right?}

\frame{
  \frametitle{About backups}

  The idea of \texttt{pg\_staging} is to maintain a \textit{staging}
  environment from production backups. If you don't have one, this tool will
  not do any good for you, but you have bigger problems than that. Supported
  backups are:

  \begin{itemize}
   \item<2-> \texttt{pg\_dump}
   \item<3-> \textit{PITR archives} (ongoing work)
  \end{itemize}
}

\section{Describing staging envs}

\frame{
  \frametitle{About backups}

  When developping \texttt{pg\_staging}, the aim is to manage a staging
  environment with:

  \begin{itemize}
   \item<2-> a dev env, with new code and old database containing
     developpers test data
   \item<3-> a prelive env with code to get in production and the most
     recent possible data from production
  \end{itemize}
}

\section{pg\_staging design \& dependancies}
\subsection{design}

\frame{
  \frametitle{Design}

  What do we want the tool to do?

  \begin{itemize}
   \item<2-> easily restore production database on staging env
   \item<3-> filtering out (\textit{historical}) data we don't need in staging
   \item<4-> allow to restore more than one version of production database
   \item<5-> allow to easily switch from one version to the other
   \item<6-> offer interactive console usage and be cron friendly too
  \end{itemize}
}

\begin{frame}[fragile]
  \frametitle{Easy restore with filtering}

  The \texttt{restore} command will \texttt{createdb}, \textit{fetch} the
  wanted backup, \textit{filter} the dump catalog, \texttt{pg\_restore}
  selected data then optionaly switch the staging env to this new database.

  \pause
  \linebreak

  \begin{example}
  \begin{verbatim}
schemas         = public, payment, utils, jdb, operations, statistiques
schemas_nodata  = sessions, archives
  \end{verbatim}
  \end{example}
\end{frame}

\frame{
  \frametitle{Dependancies}

  The following tools are used by \texttt{pg\_staging}:

  \begin{itemize}
   \item<2-> \texttt{apache} to serve the backups
   \item<3-> \texttt{pgbouncer} for database switching
   \item<4-> \texttt{postgresql-client-8.x} for dump \& restore
   \item<5-> \texttt{staging-client.sh}
   \item<6-> non-interactive \texttt{ssh}
   \item<7-> \texttt{python}
  \end{itemize}
}

\subsection{pgbouncer}

\frame{
  \frametitle{Switching databases with pgbouncer}
}

\subsection{pg\_restore}

\frame{
  \frametitle{Filtering objects from dump files}

  pg\_restore -L

  catalog \& triggers command
}

\frame{
  \frametitle{Cluster globals, init command}

  init
}

\frame{
  \frametitle{custom \texttt{SQL}}

  presql / postsql options and commands
}

\subsection{pg\_dump}

\frame{
  \frametitle{\texttt{pg\_dump} support}
}

\subsection{londiste}

\frame{
  \frametitle{londiste support}

  skipping tables we are a subscriber of, see \texttt{nodata} command

}

\section{usage and documentation}

\frame{
  \frametitle{pg\_staging notions}

  commands, config, dbname, backup date
}

\begin{frame}[fragile]
  \frametitle{Interactive or not?}

  \begin{example}
  \begin{verbatim}
#  pg_staging restore mydb today
#  pg_staging < foo.pgs
#  pg_staging
Welcome to pg_staging 0.7.
pg_staging> 
  \end{verbatim}
  \end{example}
\end{frame}

\section{distribution and status}

\begin{frame}[fragile]
  \frametitle{pgfoundry or github...}

  \texttt{pgfoundry} will host the releases when they happen.

  \linebreak
  \linebreak
  \pause

  The code is hosted at \url{http://github.com/dimitri/pg_staging}, along
  with the debian packaging. Go \texttt{git clone} and try it, we use it
  about daily.
\end{frame}

\begin{frame}[fragile]
  \frametitle{Any question?}

  \center{Now is the time to ask!}

  \linebreak
  \pause

  \begin{center}
  If you want to leave feedback, consider visiting
  \url{http://2009.pgday.eu/feedback}
  \end{center}
\end{frame}

\end{document}
