\documentclass[english]{beamer}
\usepackage[utf8,latin9]{inputenc}
%%\usepackage[latin9]{inputenc}
\usepackage[T1]{fontenc}
\usepackage{babel}

\usepackage{beamerthemesplit}
\usetheme{Warsaw}
\beamertemplatetransparentcovered

\title{Status report on modules}
\subtitle{Except we call them extensions now}
\author{Dimitri Fontaine}
\date{May, 19 2010}

\begin{document}

\frame{\titlepage}

\section{User Design}
\frame{
  \frametitle{Set the goal}

  What we're talking about:

  \begin{itemize}
   \item<1-> dump \& reload support
   \item<2-> any source language (C, SQL, PL...)
   \item<3-> procedural language as an extension
   \item<4-> custom variables
   \item<5-> versions \& dependencies
   \item<6-> \texttt{PGXS} and platform support
   \item<7-> upgrading facilities (\textit{callback})
  \end{itemize}
}

\frame{
  \frametitle{Set the goal}

  What we're \textit{NOT} talking about:

  \begin{itemize}
   \item<1-> user defined schema where to install (\texttt{pg\_extension})
   \item<2-> source level packaging
   \item<3-> \texttt{ACLs}
   \item<4-> \texttt{PGAN}
   \item<5-> OS level packaging \& distribution
  \end{itemize}
}

\subsection{2 step install syntax}

\begin{frame}[fragile]
  \frametitle{Step 1: make install}

  Support files are in \texttt{\$PGDATA/extensions/foo}, and are
  \texttt{control}, \texttt{install.sql} and \texttt{uninstall.sql},
  \texttt{foo.conf}.

\begin{example}[extensions/foo/control]
\begin{verbatim}
name = foo
version = 1.0
custom_variable_classes = 'foo'
depends  = bar (>= 1.1), baz
conflicts = bla (< 0.8)
\end{verbatim}
\end{example}

\end{frame}

\begin{frame}[fragile]
  \frametitle{Step 2: INSTALL EXTENSION}

\begin{example}[install extension]
\begin{verbatim}
 begin;
 install extension foo;
 commit;
\end{verbatim}
\end{example}

\begin{example}[drop extension]
\begin{verbatim}
 begin;
 drop extension foo [cascade];
 commit;
\end{verbatim}
\end{example}
\end{frame}

\section{Implementation}
\begin{frame}[fragile]
  \frametitle{pg\_execute\_from\_file()}

  This is the easiest part to implement, so that's done... available in git
  already. Does it make sense on its own?

\begin{example}[execute from file]
\begin{verbatim}
+Datum
+pg_execute_from_file(PG_FUNCTION_ARGS)
+{
+	text	   *filename_t = PG_GETARG_TEXT_P(0);
+
+	foreach(parsetree_item, parsetree_list)
+
\end{verbatim}
\end{example}
\end{frame}

\frame{
  \frametitle{tracking objects}
  
  We'll need to add a \textit{backend local variable} and some action on
  each and every \texttt{CREATE} command.  
}

\frame{
  \frametitle{Conclusion}

  \begin{center} 
    We need to talk.
  \end{center} 
}

\end{document}
