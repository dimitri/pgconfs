\documentclass{beamer}

\usepackage{beamerthemesplit}
\usetheme{Warsaw}

\title{PgLoader, the parallel ETL for PostgreSQL}
\author{Dimitri Fontaine}
\date{October 17, 2008}

\begin{document}

\frame{\titlepage}

\section*{Outline}
\frame{
  \frametitle{Table of contents}
  \tableofcontents
}

\section{Introduction}
\subsection{pgloader, the what?}

\frame
{
  \frametitle{ETL}

  \begin{definition}
    An \alert{ETL} process data to load into the database from a flat
    file.
  \end{definition}

  \begin{enumerate}
   \item Extract
   \item Transform
   \item Load
  \end{enumerate}
}

\frame
{
  \frametitle{pgloader's features}

  PGLoader will:

  \begin{itemize}
   \item<1-> Load CSV data
   \item<2-> Load pretend-to-be CSV data
   \item<3-> Continue loading when confronted to errors
   \item<4-> Apply user define transformation to data, on the fly
   \item<5-> Optionaly have all your cores participate into processing
  \end{itemize}
}

\section{Architecture}
\subsection{Main components}

\begin{frame}[fragile]
  \frametitle{Configuration}

  We first parse the configuration, with templating system

  \begin{example}
  \begin{verbatim}
[simple]
use_template = simple_tmpl
table        = simple
filename     = simple/simple.data
columns      = a:1, b:3, c:2
  \end{verbatim}
  \end{example}  
\end{frame}

\frame{
  \frametitle{Loading: file reading}

  PGLoader supports many input formats, even if they all look like
  \texttt{CSV}, the rough time is parsing data:

  \begin{itemize}
   \item<1-> Read files one line at a time
   \item<2-> Parse physical lines into logical lines
   \item<3-> Supports several readers
    \begin{itemize}
     \item<4-> textreader
     \item<3-> csvreader
     \item<5-> fixedreader
    \end{itemize}
  \end{itemize}
}

\frame{
  \frametitle{Processing lines}

  Parsing data is the \texttt{CPU} intensive part of the job. You could even
  have to guess where lines begin and end. \onslide<2->Then you add:

  \begin{itemize}
   \item<2-> columns restrictions
   \item<3-> columns reordering
   \item<4-> user defined columns (constants)
   \item<5-> user defined reformating modules
  \end{itemize}
}

\frame{
  \frametitle{\texttt{COPY}ing to PostgreSQL}

  This is how we do it:

  \begin{itemize}
   \item<1-> python \texttt{cStringIO} buffers
   \item<2-> configurable size (\texttt{copy\_every})
   \item<3-> using \texttt{copy\_expert()} when available (CVS)
   \item<4-> dichotomic error search
  \end{itemize}
}

\frame{
  \frametitle{Handling of erroneous data input}

  PGLoader will continue processing your input when it contains erroneous
  data.

  \begin{itemize}
   \item<1-> reject data file
   \item<2-> reject log file, containing error messages
   \item<3-> errors count in \texttt{summary}
  \end{itemize}
}

\frame{
  \frametitle{Error logging}

  PGLoader will continue processing your input when it contains erroneous
  data, \alert{and} will make it so that you know about the failures.

  \begin{itemize}
   \item<1-> log file
   \item<2-> console log level: \texttt{client\_min\_messages}
   \item<3-> logfile log level: \texttt{log\_min\_messages}
  \end{itemize}
}

\subsection{Parallel Organisation}

\frame{
  \frametitle{Why going parallel?}

  Loading is IO bound, not CPU bound, right?

  \begin{itemize}
   \item <2-> for large disks array, \textit{not so much}
   \item <3-> with complex parsing, \textit{not so much}
   \item <4-> with heavy user rewritting, \textit{not so much}
  \end{itemize}
}

\begin{frame}[fragile]
  \frametitle{Ok... How?}

  \begin{itemize}
   \item <1-> mutli-threading is easy to start with in python
   \item <2-> then you add in dequeues and semaphores (critical sections)
     and signals
   \item <3-> Giant Interpreter Lock
   \item <4-> fork() based reimplementation could be of interrest
  \end{itemize}

  \begin{example}
  \begin{verbatim}
    class PGLoader(threading.Thread):
  \end{verbatim}
  \end{example}

\end{frame}

\frame{
  \frametitle{Parallelism choices}

  Has beed asked by some \-hackers, their use cases dictated two different
  modes.
  \linebreak
  \pause

  The idea is to have a parallel pg\_restore testbed, interresting with
  large input files (100GB to several TB). PGLoader's can't compete to plain
  \texttt{COPY}, due to client\-server roundtrips compared to local file
  reading, but with some more CPUs feeding the disk array, should show up
  nice improvements.
  \linebreak
  \pause

  Testing and feeback more than welcome!
}

\begin{frame}[fragile]
  \frametitle{Round robin reader}

  Parsing is all done by a single thread for all the content.
  \linebreak
  \pause

  \texttt{N} readers are started and get each a queue where to fill this
  round data, and issue \texttt{COPY} while main reader continue parsing.
  \linebreak
  \pause

  \begin{example}
  \begin{verbatim}
[rrr]
section_threads    = 3
split_file_reading = False
  \end{verbatim}
  \end{example}
\end{frame}

\begin{frame}[fragile]
  \frametitle{Split file reader}

  The file is split into \texttt{N} blocks and there's as much pgloader
  doing the same job in parallel as there are blocks.
  \linebreak
  \pause

  \begin{example}
  \begin{verbatim}
[rrr]
section_threads    = 3
split_file_reading = True
  \end{verbatim}
  \end{example}
\end{frame}

\section{Configuration examples \& Usage}

\frame{
  \frametitle{Examples}

  PGLoader distribution comes with diverse examples, don't forget to see
  about them.  
}

\begin{frame}[fragile]
  \frametitle{simple}

  That simple:
  \linebreak
  \pause

  \begin{example}
  \begin{verbatim}
[simple]
table        = simple
filename     = simple/simple.data
format       = text
datestyle    = dmy
field_sep    = |
trailing_sep = True
columns      = *
  \end{verbatim}
  \end{example}
\end{frame}

\begin{frame}[fragile]
  \frametitle{User defined columns}

  Constant columns added at parsing time.
  \linebreak
  \pause

  Use case: adding an origin\_server\_id field depending on the file
  to get loaded, for data aggregation.
  \linebreak
  \pause

  \begin{example}
  \begin{verbatim}
   [server_A]
   file           = imports/A.csv
   columns        = b:2, d:1, x:3, y:4
   udc_c          = A
   copy_columns   = b, c, d
  \end{verbatim}
  \end{example}
\end{frame}

\begin{frame}[fragile]
  \frametitle{User defined Reformating modules}

  The basic idea is to avoid any pre-processing done with another tool
  (\texttt{sed}, \texttt{awk}, you name it).
  \linebreak 
  \onslide<2-> file has \texttt{'12131415'}
  \onslide<3-> we want \texttt{'12:13:14.15'}
  \pause

  \begin{example}
  \begin{overprint}
  
  \onslide<2>
  \begin{verbatim}
[fixed]
table           = fixed
format          = fixed
filename        = fixed/fixed.data
columns         = *
fixed_specs     = a:0:10, b:10:8, c:18:8, d:26:17
reformat        = c:pgtime:time
  \end{verbatim}

  \onslide<3>
  \begin{verbatim}
def time(reject, input):
    """ Reformat str as a PostgreSQL time """
    if len(input) != 8:
        reject.log(mesg, input)

    hour       = input[0:2]
    ...
    return '%s:%s:%s.%s' % (hour, min, secs, cents)
  \end{verbatim}
  \end{overprint}
  \end{example}
\end{frame}

\begin{frame}[fragile]
  \frametitle{The fine manual says it all}

  At \url{http://pgloader.projects.postgresql.org/} or \texttt{man pgloader}

  \begin{example}
  \begin{verbatim}
> pgloader --help
> pgloader --version
> pgloader -DTsc pgloader.conf
  \end{verbatim}
  \end{example}
\end{frame}

\section{Current status \& TODO}

\frame{
  \frametitle{TODO}

  \url{http://pgloader.projects.postgresql.org/dev/TODO.html}

  \begin{itemize}
   \item <2-> Constraint Exclusion support
   \item <3-> Reject Behaviour
   \item <4-> XML support with user defined XSLT StyleSheet
   \item <5-> Facilities
  \end{itemize}

  \onslide<6> Don't be shy and just ask for new features!
}

\frame{
  \frametitle{Resources and Users}

  pgfoundry, 1 developper, some users, no mailing list yet (no one asking
  for one), some mails sometime, seldom bug reports (fixed) \linebreak
  \pause

  Support ongoing at \texttt{\#postgresql} and \texttt{\#postgresqlfr}
  \linebreak
  \pause

  packages for \texttt{debian}, \texttt{FreeBSD}, \texttt{OpenBSD},
  \texttt{CentOS}, \texttt{RHEL} and \texttt{Fedora}.  

}

\end{document}
